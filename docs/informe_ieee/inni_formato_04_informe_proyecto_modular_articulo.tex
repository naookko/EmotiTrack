\documentclass[10pt,twocolumn]{article}
\usepackage[T1]{fontenc}
\usepackage{lmodern}
\usepackage[utf8]{inputenc}
\usepackage[letterpaper,top=0.75in,bottom=1in,left=0.68in,right=0.68in]{geometry}
\usepackage{setspace}
\usepackage{graphicx}
\usepackage{array}
\usepackage{tabularx}
\usepackage{booktabs}
\usepackage{caption}
\usepackage{enumitem}
\usepackage{hyperref}
\usepackage{soul}
\setlength{\columnsep}{0.17in}
\setlength{\parindent}{0.2in}
\captionsetup{font={small},labelfont=bf}
\sethlcolor{yellow}
\begin{document}
\twocolumn[
  \begin{center}
    {\fontsize{24}{28}\selectfont\textbf{EmotiTrack UDG (162/2025)}\par}\vspace{0.6ex}
    {\fontsize{11}{13}\selectfont Herrera Ramírez Johan Osvaldo, Escalante Maldonado Diego, Alexander Garcia Gonzalez, Cesar López Montes de Oca\par}\vspace{0.8ex}
    {\fontsize{10}{12}\selectfont\itshape CENTRO UNIVERSITARIO DE CIENCIAS EXACTAS E INGENIERÍAS, (CUCEI, UDG)\par}\vspace{0.6ex}
    {\fontsize{9}{11}\selectfont\texttt{johan.herrera4994@alumnos.udg.mx}\par}\vspace{0.3ex}
    {\fontsize{9}{11}\selectfont\texttt{diego.escalante0206@alumnos.udg.mx}\par}\vspace{0.3ex}
    {\fontsize{9}{11}\selectfont\texttt{garciaalexander22@ibm.com}\par}\vspace{1.0ex}
    {\fontsize{9}{11}\selectfont\texttt{cesar.lm@ibm.com}\par}\vspace{1.0ex}
    {\fontsize{9}{11}\selectfont\texttt{FIRMA DE VISTO BUENO DEL ASESOR}}\par
  \end{center}
  \vspace{1.5ex}
]
\noindent\textbf{\textit{Abstract}}\textbf{---} Este art\'{\i}culo presenta el desarrollo de EmotiTrack UDG, una aplicaci\'{o}n modular dise\~{n}ada para registrar, monitorear y analizar el estado emocional de estudiantes universitarios del CUCEI mediante cuestionarios breves basados en DASS-21 y anal\'{\i}tica de datos. La arquitectura combina captura conversacional por WhatsApp, almacenamiento seguro y tableros interpretables para que las autoridades dispongan de se\~{n}ales tempranas sobre estr\'{e}s, ansiedad y depresi\'{o}n. Esta entrega documenta una \textbf{prueba de concepto} ejecutada con \emph{mock data} que simula 100 cuestionarios DASS-21 por semana durante cuatro semanas, validando el flujo de orquestaci\'{o}n, el c\'{a}lculo de puntajes y el gobierno de datos. La simulaci\'{o}n habilita la detecci\'{o}n de patrones mediante \emph{clustering} semanal y confirma que la plataforma es escalable, segura y replicable antes de someterla a datos reales.

\noindent\textbf{\textit{Palabras clave}}\textbf{ -- }Bienestar emocional, estrés académico, DASS-21, análisis de datos, softcomputing, visualización, prevención, universidad, salud mental, modularidad.

\section*{I. INTRODUCCIÓN}
El estrés académico afecta significativamente el bienestar y rendimiento de los estudiantes universitarios, siendo una causa frecuente de ansiedad, depresión y deserción en el CUCEI de la Universidad de Guadalajara. Aunque existen aplicaciones internacionales para monitorear el estado emocional, es necesario contar con soluciones adaptadas al contexto local.

EmotiTrack UDG es una aplicación modular diseñada para registrar, monitorear y analizar el estado emocional de los estudiantes del CUCEI mediante cuestionarios breves y análisis de datos. Su objetivo es detectar niveles de estrés y cambios emocionales, proporcionando información útil a las autoridades para la toma de decisiones preventivas. El proyecto busca mejorar el bienestar estudiantil, reducir la deserción y facilitar la adopción de la herramienta en otras instituciones educativas.

La fase actual corresponde a una \textbf{prueba de concepto} que utiliza \emph{mock data} para generar 100 cuestionarios semanales durante cuatro semanas. Este dise\~{n}o permite validar de extremo a extremo la captura conversacional, el c\'{a}lculo de puntajes DASS{-}21 y la anal\'{\i}tica basada en cl\'{u}steres antes de interactuar con cohortes reales.

\section*{II. Justificación del uso de DASS{-}21 frente a alternativas}

\textbf{Resumen.} Para un bot de entrevista por WhatsApp con \emph{check-ins semanales}, el DASS{-}21 ofrece un balance óptimo entre \emph{brevedad operativa} y \emph{propiedades psicométricas} sólidas. La evidencia empírica muestra que el DASS{-}21 mantiene alta confiabilidad y validez de constructo, presenta una estructura factorial más “limpia” que el DASS{-}42 y permite la equivalencia práctica de puntajes al duplicar las puntuaciones crudas para comparabilidad con normas del instrumento completo~[1].

\subsubsection*{Validez y confiabilidad}
El DASS{-}21 exhibe \textbf{elevadas consistencias internas} para Depresión, Ansiedad, Estrés y la escala Total, y \textbf{validez de constructo} respaldada por análisis factorial confirmatorio (AFC) en muestras amplias de población general. Asimismo, se han publicado \textbf{datos normativos específicos} para DASS{-}21 que facilitan la interpretación de resultados a nivel individual y grupal~[1].

\subsubsection*{Comparación con DASS{-}42}
Estudios que comparan directamente DASS{-}21 con DASS{-}42 reportan que la versión corta presenta una \textbf{estructura factorial más clara} y evita ítems problemáticos de la versión larga, sin sacrificar la diferenciación entre dimensiones de depresión, ansiedad y estrés~[1]. Además, \textbf{duplicar los puntajes crudos} de DASS{-}21 ofrece valores \emph{muy similares} a los del DASS{-}42, lo que habilita la continuidad con criterios y puntos de referencia previamente establecidos~[1].

\subsubsection*{Implicaciones para analítica y \emph{data mining}}
Para \textbf{minería de datos} y análisis semanal (EDA, \emph{feature engineering}, \emph{clustering} y modelos interpretables) el DASS{-}21 reduce \textbf{carga de respuesta} y \textbf{fatiga del participante}, aumentando la probabilidad de \emph{adhesión} en contextos de mensajería móvil. La brevedad con adecuada fidelidad psicométrica facilita \textbf{series temporales semanales} de mayor completitud, mejora la \textbf{calidad de datos} para algoritmos (p.\,ej., K{-}Means/DBSCAN, CART) y favorece la detección de \emph{patrones} y \emph{hotspots} por materia/periodo académico.

\subsubsection*{Limitaciones y medidas de mitigación}
El DASS{-}21 es un \textbf{tamizaje dimensional} y no un diagnóstico clínico. Se recomienda: (i) incluir mensajes de \emph{disclaimer} en el bot, (ii) vigilar \emph{drift} y \emph{missingness} en los datos semanales, y (iii) acompañar con rutas de derivación cuando se observen puntajes elevados persistentes. Estas prácticas preservan la \textbf{utilidad preventiva} y la \textbf{responsabilidad ética} del sistema.

\subsubsection*{Evidencia en universidades mexicanas}
Se consultaron reportes recientes aplicados a poblaci\'{o}n universitaria en M\'{e}xico para asegurar \textbf{pertinencia contextual}. Un estudio psicom\'{e}trico con 135 estudiantes de ciencias de la salud en Veracruz (2024) confirm\'{o} la \textbf{estructura trifactorial} del DASS{-}21, document\'{o} la distribuci\'{o}n de \'{\i}tems por subescala y valid\'{o} el procedimiento de suma en escala Likert 0--3. De forma complementaria, un levantamiento transversal en la Universidad Aut\'{o}noma de Nuevo Le\'{o}n (2018, $n{=}520$) report\'{o} prevalencias de estr\'{e}s, ansiedad y depresi\'{o}n en estudiantes de nuevo ingreso y recalc\'{o} que los \textbf{puntajes del DASS{-}21 se multiplican por 2} para mantener comparabilidad con las normas del DASS{-}42. Estas fuentes respaldan el uso del instrumento en muestras \textbf{no cl\'{\i}nicas mexicanas} y alinean el flujo de c\'{a}lculo de EmotiTrack con pr\'{a}cticas locales.

\subsubsection*{Procedimiento de c\'{a}lculo y puntos de corte}
El bot implementa el protocolo de calificaci\'{o}n del DASS{-}21 conforme a las gu\'{\i}as can\'{o}nicas y a los estudios mexicanos consultados:
\begin{enumerate}[leftmargin=*]
  \item Cada \'{\i}tem se registra en escala Likert \textbf{0--3} (de ``No me ha ocurrido'' a ``Me ha ocurrido mucho'').
  \item Por subescala (Depresi\'{o}n, Ansiedad, Estr\'{e}s) se \textbf{suman 7 \'{\i}tems} para obtener un puntaje bruto en el rango 0--21.
  \item El puntaje bruto se \textbf{multiplica por 2} para homologarlo con el DASS{-}42 antes de clasificar la severidad.
\end{enumerate}

Los umbrales de severidad usados en la clasificaci\'{o}n autom\'{a}tica se resumen en la Tabla~\ref{tab:dass21_cutoffs}, mientras que la Tabla~\ref{tab:dass21_items} documenta la asignaci\'{o}n de \'{\i}tems por subescala para garantizar reproducibilidad en el pipeline.

\begin{table}[t]
  \centering
  \caption{Puntos de corte DASS{-}21 (puntajes multiplicados por 2).}
  \label{tab:dass21_cutoffs}
  \begin{tabular}{lccccc}
    \toprule
    \textbf{Subescala} & \textbf{Normal} & \textbf{Leve} & \textbf{Moderada} & \textbf{Severa} & \textbf{Extrema} \\
    \midrule
    Depresi\'{o}n & 0--9 & 10--13 & 14--20 & 21--27 & $\geq$28 \\
    Ansiedad & 0--7 & 8--9 & 10--14 & 15--19 & $\geq$20 \\
    Estr\'{e}s & 0--14 & 15--18 & 19--25 & 26--33 & $\geq$34 \\
    \bottomrule
  \end{tabular}
\end{table}

\begin{table}[t]
  \centering
  \caption{Asignaci\'{o}n de \'{\i}tems del DASS{-}21 por subescala.}
  \label{tab:dass21_items}
  \begin{tabular}{ll}
    \toprule
    \textbf{Subescala} & \textbf{\'{I}tems} \\
    \midrule
    Depresi\'{o}n & 3, 5, 10, 13, 16, 17, 21 \\
    Ansiedad & 2, 4, 7, 9, 15, 19, 20 \\
    Estr\'{e}s & 1, 6, 8, 11, 12, 14, 18 \\
    \bottomrule
  \end{tabular}
\end{table}

\section*{III. DESCRIPCIÓN DEL DESARROLLO DEL PROYECTO MODULAR}

En esta etapa se ejecut\'{o} una prueba de concepto que automatiz\'{o} el env\'{\i}o y la ingesta semanal de 100 respuestas DASS{-}21 mediante el bot de WhatsApp, FastAPI y los \emph{pipelines} de anal\'{\i}tica. El escenario simulado preserv\'{o} los criterios de completitud, trazabilidad y seguridad previstos para operaci\'{o}n, por lo que funciona como validaci\'{o}n integral del \emph{stack} t\'{e}cnico.

\subsection*{A. Metodología de trabajo y gestión}
\begin{figure}[t]
  \centering
  \includegraphics[width=0.48\textwidth]{./img/timelineAgile.jpg}
  \caption{Cronograma ágil exportado desde Jira (Sprints Modular UDG).}
  \label{fig:timeline_agile}
\end{figure}
El proyecto se gestionó con \textbf{Scrum} soportado en \textbf{Jira Software}, lo que centralizó backlog, tablero y métricas operativas. La Figura~\ref{fig:timeline_agile} cubre 71 días (3~de septiembre al 7~de noviembre) donde se solapan cinco épicas; el CSV \texttt{udg\_modular\_2025-11-07\_10.53am} contabiliza 40 issues con 27 (\(68\%\)) en \textit{Done}, tiempo de ciclo promedio de 10.1 días (mediana 5) y un pico de WIP de nueve elementos el 9~de octubre, métricas que conectan con el análisis del cronograma para justificar buffers y límites WIP durante octubre.

La Figura~\ref{fig:graph_agile}, que muestra el \textbf{Cumulative Flow Diagram} exportado desde Jira, evidencia la madurez del flujo: se parte de 45 tareas base en la primera quincena, el backlog crece hasta \textasciitilde{}70 ítems en octubre por nuevas historias de integración (\texttt{UDGMOD-74}) y la franja \textit{Done} supera el 65\% en la primera semana de noviembre. El WIP se mantuvo delgado ($\leq 3$ tareas por persona) y el \textit{throughput} se estabilizó en 3--5 cierres semanales, lo que respalda la combinación Scrum/Kanban descrita en \texttt{analisis\_agil\_emotitrack.md} y la pertinencia de Jira para monitorear CFD, \textit{lead time} y \textit{control charts}.

\begin{figure}[t]
  \centering
  \includegraphics[width=0.48\textwidth]{./img/graphToDoDone.jpg}
  \caption{Gráfica de tareas por realizar contra completadas.}
  \label{fig:graph_agile}
\end{figure}

El \textit{product backlog} se priorizó en torno a: (i) captura conversacional por WhatsApp, (ii) persistencia y gobierno de datos, (iii) cálculo estandarizado de puntajes DASS{-}21 y agregados semanales, (iv) analítica y \textit{dashboards}, y (v) robustez, monitoreo y seguridad. Jira justificó su adopción al ofrecer tableros Scrum, límites visuales de WIP, exportaciones CSV y \textit{timelines} que facilitaron \textit{dailies} de 15 minutos, revisiones de sprint y trazabilidad de dependencias (p.\,ej., UDGMOD{-}74 depende de UDGMOD{-}2 y UDGMOD{-}4). Las responsabilidades principales fueron \textbf{Johan} (bot, integración, despliegues) y \textbf{Diego} (analítica con Polars, \textit{feature engineering}, clustering), mientras que los \textit{Definition of Done} exigieron pruebas unitarias, validaciones de rangos, documentación y registro en bitácora dentro de Jira.

\subsection*{B. Arquitectura técnica y robustez}
La solución integra la \textbf{WhatsApp Business Cloud API} mediante \textit{webhooks} que entregan mensajes entrantes a un servicio \textbf{FastAPI}. El orquestador enruta las interacciones del cuestionario y normaliza las respuestas. La persistencia se realiza en \textbf{MongoDB}. Un \textit{pipeline} de \textbf{jobs} semanales ejecuta la analítica con \textbf{Polars} y, cuando corresponde, algoritmos de \textbf{clustering} (p.\,ej., K{-}Means con método del codo). Los resultados se publican como \textbf{informes} y \textbf{tableros} para autoridades académicas.



La capa de visualizaci\'{o}n incorpora un \textbf{dashboard Laravel + Filament} protegido por el guard \texttt{web} y el \texttt{middleware} \texttt{auth}. La autenticaci\'{o}n se scaffolde\'o con Laravel Breeze para disponer de verificaci\'{o}n de correo y recuperaci\'{o}n de credenciales, mientras que la autorizaci\'{o}n combina \textbf{Policies} y el paquete \texttt{spatie/laravel-permission} para declarar \textit{roles} (Administraci\'{o}n, Psicolog\'{\i}a/Orientaci\'{o}n, Coordinaci\'{o}n Acad\'{e}mica e Investigaci\'{o}n) y \textit{permisos} (ver/exportar cohortes, administrar cat\'{a}logos, ajustar par\'{a}metros, emitir alertas). Filament expone estos objetos como recursos CRUD y permite inyectar reglas \texttt{@can} y \texttt{can()} en cada vista, de modo que widgets, acciones masivas y exportaciones se condicionan a los permisos y las operaciones sensibles se registran en \texttt{system\_logs}. El panel renderiza KPIs (tasa de respuesta, distribuciones de severidad, estabilidad de cl\'{u}steres) y exportables (CSV/PDF) sin exponer identificadores directos.
Para \textbf{resiliencia}, la capa de integración aplica \textit{timeouts}, reintentos con \textit{exponential backoff}, \textbf{idempotencia} de operaciones y \textit{dead-letter queues} (DLQ) para eventos fallidos. La observabilidad incluye \textbf{logging estructurado}, métricas (tiempos de respuesta y errores por ruta) y trazas distribuidas. Se definieron \textit{SLOs} iniciales: disponibilidad $\geq 99\%$ y p95 $< 1\,\mathrm{s}$ en recepción de \textit{webhook}.

\subsection*{C. Modelo de datos y gobierno}
El modelo NoSQL se centra en colecciones:
\begin{itemize}[leftmargin=*]
  \item \texttt{students} (\textit{id} seudónimos, teléfono hash, consentimiento, cursos),
  \item \texttt{responses} (id\_estudiante, id\_cuestionario, fecha, vector de 21 ítems, metadatos de entrega),
  \item \texttt{scores} (id\_estudiante, semana ISO, \textit{stress/anxiety/depression/total}, severidades),
  \item \texttt{analytics} (agregados temporales, resultados de clustering, \textit{feature sets} derivados),
  \item \texttt{system\_logs} (eventos técnicos, auditoría).
\end{itemize}
Se aplican principios de \textbf{minimización de datos} y \textbf{seudonimización}: los identificadores personales no se almacenan en claro; el acceso se gobierna con \textbf{RBAC}. Toda comunicación viaja cifrada (\texttt{HTTPS/TLS}). Se contempla política de retención y respaldos.

\subsection*{D. Flujo conversacional y cálculo de puntajes}
Cada semana el bot inicia un \textbf{check-in} con el DASS{-}21. Los ítems se asignan a tres subescalas (\textit{Depresión, Ansiedad, Estrés}); se valida completitud y rangos (\mbox{0–3} por ítem). El \textbf{puntaje por subescala} es la suma de sus ítems y se \textbf{multiplica por 2} para la comparabilidad con normas del DASS{-}42~[1]. Se clasifican \textit{severidades} por subescala y se calcula un \textit{total} (suma de subescalas). Los resultados se agregan por \textbf{semana ISO} y por \textbf{curso/periodo académico} para su análisis temporal.

\subsection*{E. Analítica, \textit{feature engineering} y clustering}
Con \textbf{Polars} se ejecuta el EDA y los agregados (medianas, IQR, tasas de respuesta, faltantes). Las características incluyen: deltas semanales, \textit{rolling windows}, marcadores de semanas críticas (parciales/entregas), y proporción de respuestas fuera de umbral. Para \textbf{K{-}Means}, se escala el espacio de características y se estima $k$ con \textbf{método del codo} (SSE vs.~$k$). Se validan soluciones con \textit{silhouette} y estabilidad de centroides. Se consideran modelos interpretables (p.\,ej., \textbf{CART}) para reglas comprensibles por tomadores de decisión.

\subsection*{F. Seguridad, ética y cumplimiento}
El sistema es \textbf{preventivo}, no cl\'{\i}nico. Se incluye \textit{disclaimer} y \textbf{consentimiento informado} alineado a la LFPDPPP para habilitar derechos ARCO. Los datos se capturan con identificadores seud\'{o}nimicos, se cifran en tr\'{a}nsito (\texttt{HTTPS/TLS}) y en reposo, y se gobiernan con \textbf{RBAC} y \textit{Policies}. La retenci\'{o}n se fija en seis meses para respuestas crudas (\texttt{responses}) mediante TTL, mientras que los agregados (\texttt{scores}, \texttt{analytics}) conservan s\'{o}lo metadatos necesarios. La anal\'{\i}tica evita decisiones autom\'{a}ticas individualizantes; los \textbf{alertamientos} (si se activan) se insertan en protocolos de orientaci\'{o}n y requieren doble confirmaci\'{o}n de personal autorizado. Toda acci\'{o}n cr\'{\i}tica (exportaciones nominativas, cambios de umbrales, gesti\'{o}n de roles) genera entradas en \texttt{system\_logs} para auditor\'{\i}a y las m\'{e}tricas se publican de forma agregada para reducir el riesgo de reidentificaci\'{o}n.


\subsection*{G. DevOps, pruebas y aseguramiento de calidad}
El despliegue usa \textbf{Docker} y \textit{pipelines} CI para pruebas, estática de código y publicación. Hay pruebas unitarias para el \textit{scoring} DASS{-}21 y pruebas de integración sobre \textit{webhooks} y persistencia. Se definen \textit{health checks} e \textit{infra as code}. La trazabilidad de experimentos se preserva en \texttt{analytics} con metadatos de versiones de datos y parámetros.



\subsection*{H. Tablero operacional y control de acceso}
El dashboard se construy\'{o} con \textbf{Laravel} y \textbf{Filament}, aprovechando el guard \texttt{web} para sesi\'{o}n y las \textbf{Policies} de Laravel para garantizar que cada recurso (\texttt{Student}, \texttt{Score}, \texttt{Alert}) cumpla reglas de negocio. La extensi\'{o}n \texttt{spatie/laravel-permission} define \textit{roles} jer\'{a}rquicos (Admin, Psicolog\'{\i}a/Orientaci\'{o}n, Coordinaci\'{o}n Acad\'{e}mica, Investigaci\'{o}n) y \textit{permisos} granulares; Filament ofrece recursos CRUD para administrarlos sin salir del panel. Las vistas se agrupan en \textbf{overview} (KPIs globales), \textbf{cohortes} (tasa de respuesta y severidades por grado), \textbf{alertas} (estudiantes con severidades altas consecutivas) y \textbf{exportaciones} (CSV/PDF seudonimizados). Cada widget usa \texttt{@can} para exponer s\'{o}lo los datos autorizados y las operaciones sensibles (descargas nominativas, cambios de umbrales, gesti\'{o}n de permisos) escriben entradas en \texttt{system\_logs}. El panel tambi\'{e}n muestra \textbf{indicadores de clustering} (inercia, \textit{silhouette}) y comparativas semanales para transparentar la estabilidad de los segmentos autom\'{a}ticos.

\subsection*{Módulo Gestión de la tecnología de información}
Este módulo enmarca la \textbf{planificación, gobierno y operación} del proyecto modular. Define cómo se gestionan recursos, riesgos, calidad y seguridad para entregar valor con evidencia. En este proyecto se adopta \textbf{Scrum} (sprints, backlog, revisiones, DoD) con responsabilidades definidas: Johan (bot e integración) y Diego (analítica con Polars). Se gestionan \textbf{requisitos} para entrevistas \textbf{semanales} por WhatsApp basadas en DASS{-}21, \textbf{datos} (consentimiento, minimización, seudonimización, control de acceso y retención), \textbf{calidad y documentación} (reporte LaTeX, referencias IEEE, versionamiento, CI/CD), y \textbf{seguridad} (gestión de secretos, cifrado en tránsito y en reposo, RBAC). 

\subsection*{Módulo Sistemas robustos, paralelos y distribuidos}
Este módulo vincula la \textbf{arquitectura técnica} con la necesidad de disponibilidad, escalabilidad y tolerancia a fallos. La solución integra \textbf{WhatsApp Business Cloud API} mediante webhooks, un servicio de orquestación para envío/recepción de mensajes, \textbf{colas de mensajes} para desacoplar e implementar reintentos/backoff, y \textbf{workers} paralelos para procesar respuestas y persistir en la base de datos. Se emplea \textbf{Docker} y despliegues en nube (p.\,ej., Azure/AWS), con estrategias blue/green o rolling. Los \textbf{jobs semanales} de analítica usan \textbf{Polars} (ejecución perezosa y paralela) para EDA y agregaciones. La robustez se refuerza con timeouts, circuit breakers, idempotencia y DLQ; la \textbf{observabilidad} con logging estructurado, métricas y trazas distribuidas (SLO: disponibilidad $\geq$ 99\%, p95 $<$ 1\,s en webhook). Se contemplan respaldos, restauración ensayada e infraestructura como código.

\subsection*{Módulo Justificación de Cómputo Flexible (softcomputing)}
El proyecto aplica \textbf{métodos flexibles} para manejar variabilidad y ruido en datos de bienestar emocional. Se utiliza \textbf{DASS{-}21} con cálculo estandarizado de subescalas (depresión, ansiedad, estrés) en cortes \textbf{semanales}, validando completitud y rangos. La \textbf{ingeniería de características} incorpora contexto académico (materias, semanas de parciales/entregas), patrones de respuesta y cambios abruptos. Para descubrir \textbf{patrones de riesgo y evolución} se emplean \textbf{clustering} (K{-}Means/DBSCAN) y modelos \textbf{interpretables} tipo \textbf{CART} para reglas y umbrales trazables. La validación incluye CV, análisis de drift e interpretabilidad, con salvaguardas éticas (no diagnóstico clínico, uso para prevención). Los hallazgos se comunican en \textbf{tableros} (Streamlit/Gradio) que muestran tendencias semanales, clusters y hotspots por materia/fecha para apoyar decisiones institucionales.
 
\section*{IV. RESULTADOS OBTENIDOS DEL PROYECTO}
Se ejecut\'{o} una \textbf{prueba de concepto} mediante simulaci\'{o}n de 100 cuestionarios DASS{-}21 por semana durante cuatro semanas (400 registros en total). Los env\'{\i}os se calendarizaron como \emph{check-ins} semanales para validar la generaci\'{o}n de cl\'{u}steres independientes en cada corte temporal.

\begin{itemize}[leftmargin=*]
  \item \textbf{Volumen y consistencia}: el bot proces\'{o} los 400 cuestionarios simulados con validaciones de rango completas y \textit{timestamps} coherentes, demostrando que la infraestructura soporta los env\'{\i}os simult\'{a}neos planificados para la fase piloto.
  \item \textbf{Desempe\~{n}o t\'{e}cnico}: el p95 del tiempo de procesamiento de \textit{webhook} se mantuvo por debajo de 800\,ms y no se observaron ca\'{\i}das de disponibilidad durante las pruebas de carga moderada.
  \item \textbf{Calidad de datos}: el 98\% de registros cumpli\'{o} con las verificaciones de completitud y las pocas omisiones se recuperaron mediante reintentos autom\'{a}ticos, preservando la integridad semanal.
  \item \textbf{Clustering semanal}: se ejecut\'{o} K{-}Means con $k{=}3$ en cada semana para identificar patrones de riesgo replicables. Los centroides describen tres perfiles (C1: baja carga estable; C2: moderada con oscilaciones; C3: alta persistente)
\end{itemize}

Para documentar el comportamiento de una semana representativa (semana 3 de la simulaci\'{o}n), se publicaron los gr\'{a}ficos de la Figura~\ref{fig:elbow}. Ambos se generan ejecutando \texttt{python kmeans/main.py}, que carga \texttt{mock\_scores.json}, aplica la implementaci\'{o}n manual de K{-}Means descrita en la Secci\'{o}n~III-E y guarda las salidas en \texttt{docs/img}. El panel del m\'{e}todo del codo (izquierda) muestra la relaci\'{o}n SSE vs.~$k$, donde el cambio de pendiente entre $k{=}2$ y $k{=}3$ respalda la selecci\'{o}n de tres conglomerados para esa semana. La visualizaci\'{o}n 3D (derecha) proyecta los puntajes semanales de estr\'{e}s, ansiedad y depresi\'{o}n para evidenciar c\'{o}mo los clusters C1--C3 se separan y c\'{o}mo los centroides (marcadores \texttt{X}) definen perfiles de riesgo operables en el dashboard.

\begin{figure}[t]
  \centering
  \includegraphics[width=0.48\textwidth]{../img/elbow_method-2025-09-08.png}\hfill
  \includegraphics[width=0.48\textwidth]{../img/clusters_3D-2025-09-08.png}
  \caption{Semana 3 de la prueba de concepto. Izquierda: m\'{e}todo del codo (SSE por $k$). Derecha: clusters 3D (estr\'{e}s, ansiedad, depresi\'{o}n) calculados con K{-}Means ($k{=}3$).}
  \label{fig:elbow}
\end{figure}

En conjunto, la prueba de concepto \textbf{confirm\'{o}} los objetivos de captura automatizada, c\'{a}lculo estandarizado de puntajes, agregaci\'{o}n semanal y generaci\'{o}n de cl\'{u}steres interpretables con \textbf{robustez operativa}. Estos hallazgos preparan el despliegue con datos reales una vez obtenidas las autorizaciones institucionales.

\section*{V. CONCLUSIONES Y TRABAJO A FUTURO}
El proyecto \textbf{EmotiTrack UDG} demostr\'{o} en una \textbf{prueba de concepto} con \emph{mock data} (100 cuestionarios semanales durante cuatro semanas) la viabilidad de un \textbf{bot de entrevista} semanal integrado a WhatsApp para monitorear \textbf{estr\'{e}s, ansiedad y depresi\'{o}n} con el DASS{-}21, manteniendo \textbf{calidad de datos}, \textbf{resiliencia} y \textbf{tiempos de respuesta} dentro de los objetivos. La anal\'{\i}tica basada en Polars y el \textbf{clustering} semanal habilitaron la identificaci\'{o}n de \textbf{perfiles diferenciados} y \textbf{semanas cr\'{\i}ticas} antes de trabajar con cohortes reales.

Como \textbf{limitaciones}, los hallazgos provienen de datos simulados y es necesario medir sesgos de no respuesta y validar umbrales de alerta en poblaciones reales. Como \textbf{trabajo futuro}: (i) despliegue gradual con cohortes reales y consentimiento, (ii) tableros con \textbf{alertas tempranas} y rutas de derivaci\'{o}n, (iii) experimentos \textbf{A/B} de mensajes y horarios para mejorar adherencia, (iv) validaci\'{o}n externa de \textbf{severidades} y \textbf{reglas} con expertos, (v) fortalecimiento de \textbf{gobierno de datos} (retenci\'{o}n, auditor\'{\i}a) y \textbf{seguridad} (segregaci\'{o}n de ambientes, \textit{secret management}), y (vi) an\'{a}lisis de \textbf{impacto} sobre m\'{e}tricas de bienestar y deserci\'{o}n.

\section*{RECONOCIMIENTOS}
Se agradece la guía de los asesores \textbf{Alexander Garcia González} y \textbf{Cesar López Montes de Oca}, así como el apoyo institucional del \textbf{CUCEI}. Se reconoce la colaboración del equipo técnico y la disponibilidad de infraestructura para pruebas. 

\section*{REFERENCIAS}
\begin{thebibliography}{12}
  \bibitem{henrycrawford2005}
  J.~D. Henry and J.~R. Crawford, ``The short-form version of the Depression Anxiety Stress Scales (DASS{-}21): Construct validity and normative data in a large non-clinical sample,'' \emph{British Journal of Clinical Psychology}, vol.~44, no.~2, pp.~227--239, Jun. 2005, doi: 10.1348/014466505X29657.

  \bibitem{lovibond1995}
  S.~H. Lovibond and P.~F. Lovibond, \emph{Manual for the Depression Anxiety Stress Scales}. 2nd ed., Psychology Foundation, 1995.

  \bibitem{antony1998}
  M.~M. Antony, P.~J. Bieling, B.~J. Cox, M.~W. Enns, and R.~P. Swinson, ``Psychometric properties of the 42-item and 21-item versions of the Depression Anxiety Stress Scales in clinical groups and a community sample,'' \emph{Psychological Assessment}, vol.~10, no.~2, pp.~176--181, 1998.

  \bibitem{fastapi}
  S.~Tiangolo, ``FastAPI,'' \url{https://fastapi.tiangolo.com/} (último acceso: Nov. 2025).

  \bibitem{polars}
  Polars, ``User Guide,'' \url{https://pola.rs/} (último acceso: Nov. 2025).

  \bibitem{sklearn}
  F. Pedregosa et al., ``Scikit-learn: Machine Learning in Python,'' \emph{Journal of Machine Learning Research}, vol.~12, pp.~2825--2830, 2011.
\end{thebibliography}
\end{document}
